\documentclass{article}
% common styling and macros shared by all proof files

\usepackage[top=1in, right=1in, left=1in, bottom=1.5in]{geometry}

\usepackage{amsmath,amsthm,amsfonts,amssymb,amscd}
\usepackage{listings}
\usepackage{hyperref}
\usepackage{xcolor}
\usepackage{xr}

\usepackage{enumerate} 
\usepackage{physics}
\usepackage{fancyhdr}
\usepackage{hyperref}
\usepackage{graphicx}
\usepackage{tcolorbox}
\usepackage{catchfile}
\usepackage{pdftexcmds}
\usepackage[T1]{fontenc}

% hyperref
\hypersetup{
  colorlinks=true,
  linkcolor=blue,
  linkbordercolor={0 0 1}
}

% \contrib macro to indicate inclusion in "contrib".
\usepackage{tcolorbox}
\newtcolorbox{warn_box}{colback=red!5!white,colframe=red!75!black}
\newcommand{\contrib}{{\begin{warn_box}This proof resides in \textbf{``contrib''} because it has not completed the vetting process.\end{warn_box}}} 
\newcommand{\floatingPoint}{{\begin{warn_box}This implementation is susceptible to floating-point vulnerabilities.\end{warn_box}}} 

% asOfCommit macro to version a code dependency. Arguments:
%    #1: relative path to file you are dependent on
%    #2: commit hash it was last edited. If outdated, this should be the second hash in the footnoote. Otherwise,
%            git log -n 1 --pretty=format:%h -- path/to/file.rs
\makeatletter
\ifnum\pdf@shellescape=1
   % "private" command that builds a link to a blob
  \newcommand{\linkOpendpBlob}[3]{%
    \href{https://github.com/opendp/opendp/blob/#1/#2#3}{\path{#3} at commit #1}}

  % latex macro expansion has a separate phase for \input evaluation
  %     immediately evaluate a command to write a temp file to ./out containing the current directory
  \immediate\write18{[ ! -f out/cwd.txt ] && (mkdir -p out && git rev-parse --show-prefix | sed "s|_|\@backslashchar\@backslashchar\@backslashchar_|g" > out/cwd.txt)}
  %     ...and then retrieve the current working directory by loading the temp file
  \CatchFileDef\GitWorkingDir{out/cwd.txt}{\endlinechar=-1}

  % command for building the (up to date) or (outdated) status
  \newcommand{\fileStatus}[2]{%
  \setbox0=\hbox{\input|"git --no-pager log -n1 --pretty='\@percentchar H' #1 | grep -E '^#2.*'"\unskip}\ifdim\wd0=0pt
        (outdated\footnote{See new changes with \texttt{git diff #2..\input|"git --no-pager log -n1 --pretty='\@percentchar h' #1" \GitWorkingDir\path{#1}}})\else
        (up to date)\fi
  }

  \newcommand{\asOfCommit}[2]{%
      % permalink the target
      \linkOpendpBlob{#2}{\GitWorkingDir}{#1}
      % conditionally add (outdated) or (up to date) depending on matching commit hash
      \fileStatus{#1}{#2}%
  }
\else
  % simplified command if shell-escape not enabled
  \newcommand{\asOfCommit}[2]{#1 at commit #2 (unknown status\footnote{Shell-escape is not enabled. Enable \texttt{--shell-escape} to check if this proof is up-to-date with the code.})}
\fi
\makeatother

% \vettingPR macro to link a PR. Arguments:
%    #1: PR number
\newcommand{\vettingPR}[1]{\href{https://github.com/opendp/opendp/pull/#1}{Pull Request \##1}}

% for links to rustdoc items in OpenDP. Arguments:
%    #1: path to item, and designation as trait, struct, fn, etc.
%    #2: item name
\makeatletter
\ifnum\pdf@shellescape=1
  % latex macro expansion has a separate phase for \input evaluation
  %     immediately evaluate a command to write a temp file to ./out containing the base path
  \immediate\write18{[ ! -f out/rustdoc.txt ] && mkdir -p out && ([ -z `kpsewhich --var-value OPENDP_RUSTDOC_PORT` ] && echo "https://docs.rs/opendp/`head -n 1 \@backslashchar`git rev-parse --show-toplevel\@backslashchar`/VERSION | sed 's|.*-dev.*|latest|g'`" || echo "http://localhost:`kpsewhich --var-value OPENDP_RUSTDOC_PORT`") > out/rustdoc.txt}
  %     ...and then retrieve the base path by loading the temp file
  \CatchFileDef\OpenDPRustdocBase{out/rustdoc.txt}{\endlinechar=-1}
\else
  % if shell commands are not enabled, just claim latest
  \newcommand{\OpenDPRustdocBase}{https://docs.rs/opendp/latest}
\fi
\makeatother
\newcommand{\rustdoc}[2]{\href{\OpenDPRustdocBase/opendp/#1.#2.html}{\texttt{#2}}}

% for links to external dependencies. Arguments:
%    #1: crate name
%    #2: path to item, and designation as trait, struct, fn, etc.
%    #3: item name
\newcommand{\docsrs}[3]{\href{https://docs.rs/#1/latest/#1/#2.#3.html}{\texttt{#3}}}

% minted (pseudocode)
\definecolor{codegreen}{rgb}{0,0.6,0}
\definecolor{codegray}{rgb}{0.5,0.5,0.5}
\definecolor{codepurple}{rgb}{0.58,0,0.82}
\definecolor{backcolour}{rgb}{0.95,0.95,0.92}

\lstdefinestyle{mystyle}{
    backgroundcolor=\color{backcolour},   
    commentstyle=\color{codegreen},
    keywordstyle=\color{magenta},
    numberstyle=\tiny\color{codegray},
    stringstyle=\color{codepurple},
    basicstyle=\ttfamily\footnotesize,
    breakatwhitespace=false,         
    breaklines=true,                 
    captionpos=b,                    
    keepspaces=true,                 
    numbers=left,                    
    numbersep=5pt,                  
    showspaces=false,                
    showstringspaces=false,
    showtabs=false,                  
    tabsize=2
}

\lstset{style=mystyle}

% common commands
\theoremstyle{definition}
\newtheorem{theorem}{Theorem}[section]
\newtheorem{lemma}[theorem]{Lemma}
\newtheorem{definition}[theorem]{Definition}
\newtheorem{warning}{Warning}
\newtheorem{corollary}{Corollary}
\newtheorem{proposition}{Proposition}
\newtheorem{remark}{Remark}
\newtheorem{observation}{Observation}
\newtheorem{note}{Note}

\newcommand{\vicki}[1]{{ {\color{olive}{(vicki)~#1}}}}
\newcommand{\hanwen}[1]{{ {\color{purple}{(hanwen)~#1}}}}
\newcommand{\zach}[1]{{ {\color{red}{(zach)~#1}}}}

\newcommand{\MultiSet}{\mathrm{MultiSet}}
\newcommand{\len}{\mathrm{len}}
\newcommand{\din}{\texttt{d\_in}}
\newcommand{\dout}{\texttt{d\_out}}
\newcommand{\T}{\texttt{T} }
\newcommand{\F}{\texttt{F} }
\newcommand{\Map}{\texttt{Map}}
\newcommand{\X}{\mathcal{X}}
\newcommand{\Y}{\mathcal{Y}}
\newcommand{\True}{\texttt{True}}
\newcommand{\False}{\texttt{False}}
\newcommand{\clamp}{\texttt{clamp}}
\newcommand{\function}{\texttt{function}}
\newcommand{\float}{\texttt{float }}
\newcommand{\questionc}[1]{\textcolor{red}{\textbf{Question:} #1}}


\newcommand{\validTransformation}[2]{%
  \begin{theorem}
  For every setting of the input parameters #1 to #2 such that the given preconditions
  hold, #2 raises an exception (at compile time or run time) or returns a valid transformation. A valid transformation has the following properties:
  \begin{enumerate}
      \item \textup{(Appropriate output domain).} 
      For every element $x$ in \texttt{input\_domain}, $\function(x)$ is in \texttt{output\_domain} or raises a data-independent runtime exception.
      
      \item \textup{(Stability guarantee).} 
      For every pair of elements $x, x'$ in \texttt{input\_domain} and for every pair $(\din, \dout)$, 
      where \din\ has the associated type for \texttt{input\_metric} and \dout\ has the associated type for \\ \texttt{output\_metric}, 
      if $x, x'$ are \din-close under \texttt{input\_metric}, $\texttt{stability\_map}(\din)$ does not raise an exception,
      and $\texttt{stability\_map}(\din) \leq \dout$, 
      then $\function(x), \function(x')$ are $\dout$-close under \texttt{output\_metric}.
  \end{enumerate}
  \end{theorem}
}


\newcommand{\validMeasurement}[2]{%
  \begin{theorem}
  For every setting of the input parameters #1 to #2 such that the given preconditions
  hold, #2 raises an exception (at compile time or run time) or returns a valid measurement. A valid measurement has the following property:
  \begin{enumerate}
      \item \textup{(Privacy guarantee).}
      For every pair of elements $x, x'$ in \texttt{input\_domain} and for every pair $(\din, \dout)$,
      where \din\ has the associated type for \texttt{input\_metric} and \dout\ has the associated type for \\ \texttt{output\_measure},
      if $x, x'$ are \din-close under \texttt{input\_metric}, $\texttt{privacy\_map}(\din)$ does not raise an exception,
      and $\texttt{privacy\_map}(\din) \leq \dout$,
      then $\function(x), \function(x')$ are $\dout$-close under \texttt{output\_measure}.
  \end{enumerate}
  \end{theorem}
}


\title{\texttt{fn make\_select\_private\_candidate}}
\author{Michael Shoemate}
\date{}

\begin{document}

\maketitle

\contrib
Proves soundness of \rustdoc{combinators/fn}{make\_select\_private\_candidate} in \asOfCommit{mod.rs}{f5bb719}.

\texttt{make\_select\_private\_candidate} returns a Measurement that returns a release from \texttt{measurement} whose score is above \texttt{threshold},
or may fail and return nothing.

\section{Hoare Triple}
\subsection*{Precondition}
\subsubsection*{Compiler-verified}
\begin{itemize}
    \item Generic \texttt{DI} (input domain) is a type with trait \rustdoc{core/trait}{Domain}. 
    \item Generic \texttt{MI} (input metric) is a type with trait \rustdoc{core/trait}{Metric}. 
    \item \rustdoc{core/trait}{MetricSpace} is implemented for \texttt{(DI, MI)}. Therefore \texttt{MI} is a valid metric on \texttt{DI}.
    \item Argument \texttt{measurement} is a measurement whose output metric is \rustdoc{measures/struct}{MaxDivergence}, and releases a tuple \texttt{(f64, TO)}, where \texttt{TO} is some arbitrary type.
    \item Argument \texttt{stop\_probability} is of type \texttt{f64}.
    \item Argument \texttt{threshold} is of type \texttt{f64}.
\end{itemize}

\subsubsection*{User-verified}
None

\subsection*{Pseudocode}
\lstinputlisting[language=Python,firstline=2,escapechar=|]{./pseudocode/make_select_private_candidate.py}

\subsection*{Postcondition}
\validMeasurement{\texttt{(measurement, stop\_probability, threshold, DI, MI, TO)}}{\texttt{make\_select\_private\_candidate}}

\section{Proofs}
This section shows that the pseudocode implements Theorem 3.1 in \cite{liu2018privateselectionprivatecandidates}, where 
\texttt{stop\_probability} is $\gamma$ and
\texttt{threshold} is $\tau$.

\begin{theorem}
    \label{algorithm-equivalence}
    The function pseudocode is equivalent to Algorithm 1 in \cite{liu2018privateselectionprivatecandidates}.
\end{theorem}

\begin{proof}
    \texttt{scale} is $-1 / \ln(1 - \gamma)$, with arithmetic conservatively rounded down.
    The rounding down corresponds to higher effective stop probability.
    A higher effective stop probability results in a lower effective privacy usage $\epsilon_0$.
    This means the advertised privacy loss is a conservative overestimate.

    Since $\gamma$ is restricted to $[0, 1)$ by \ref{stop-prob}, scale is non-negative.
    Therefore in \ref{sample-geometric}, 
    the precondition of \rustdoc{traits/samplers/cks20/fn}{sample\_geometric\_exp\_fast} is satisfied.
    In Algorithm 1, the mechanism is invoked once before potentially terminating, so one is added to the sample.
    \texttt{remaining\_iterations}, a sample from the shifted geometric distribution, 
    is then equivalent to a count of the number of coin flips made until sampling one heads,
    as is used in Algorithm 1.

    We then only run as many iterations as has been sampled, as in Algorithm 1.
    In the case where $T$ is infinity and $\gamma$ is zero, then the algorithm only terminates when score exceeds threshold.

    If the measurement releases a score of NaN, then this is effectively treated as a score below negative infinity, 
    as NaN will never compare greater than the threshold.

    Otherwise, the rest of the algorithm is evidently equivalent.
\end{proof}

\begin{theorem}
    \label{theorem-equivalence}
    \texttt{make\_select\_private\_candidate} is consistent with Theorem 3.1 of \cite{liu2018privateselectionprivatecandidates}.
\end{theorem}

\begin{proof}
    Since there is no limit on iterations, $\epsilon_0$ is zero.
    By \ref{algorithm-equivalence}, the function is equivalent to Algorithm 1, so we can claim parts a-e of Theorem 3.1.
    Since \texttt{measurement} is a valid measurement, then we know that it satisfies $\epsilon_1$-DP.
    The privacy map returns $2\epsilon_1 + \epsilon_0$ with arithmetic rounded conservatively up,
    which is consistent with part b in Theorem 3.1.
\end{proof}

\textbf{(Privacy guarantee.)} 
By \ref{theorem-equivalence}, assuming correctness of Theorem 3.1 part b in \cite{liu2018privateselectionprivatecandidates},
then for every pair of elements $x, x' \in \texttt{input\_domain}$ and every $d_{MI}(x, x') \le \din$ with $\din \ge 0$, 
if $x, x'$ are $\din$-close then $\function(x), \function(x')$ are $\texttt{privacy\_map}(\din)$-close under $\texttt{output\_measure}$ (the Max-Divergence).

\bibliographystyle{plain}
\bibliography{references.bib}

\end{document}

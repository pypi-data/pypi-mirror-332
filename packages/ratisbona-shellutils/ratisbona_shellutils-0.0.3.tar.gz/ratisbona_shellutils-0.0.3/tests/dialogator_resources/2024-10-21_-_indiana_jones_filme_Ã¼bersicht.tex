\documentclass[a4paper]{article}
\usepackage[ngerman]{babel}
\usepackage[dvipsnames]{xcolor}
\usepackage[export]{adjustbox} % loads also graphicx
\usepackage{amsmath}
\usepackage{emoji}
\usepackage{fontspec}
\usepackage{dialogue}
\usepackage{wasysym}
\usepackage{minted}
\usepackage{cprotect}
\usepackage[colorlinks=true, allcolors=TealBlue]{hyperref}
\usepackage{longtable,tabu}
\setlength{\parindent}{0ex}
\setlength{\parskip}{1ex}
\setminted{breaklines=true, breakanywhere=true}

\newcommand{\uliji}[1]{
  {\setmainfont{Noto Color Emoji}[Renderer=Harfbuzz]{#1}}
}


\author{Ulrich Schwenk}
\title{test gpt}

\begin{document}
\setlength{\tabulinesep}{6pt}
\maketitle
\newpage

\begin{dialogue}
\speak{system} \leavevmode\\
\end{dialogue}




\section*{2024-10-21}
\begin{dialogue}
\speak{user} \leavevmode\\
Hi,

wie heißen die Indiana-Jones Filme auf Deutsch und auf Englisch? Bitte liefere eine Tabelle. Füge auch das Erscheinungs-Jahr des Filmes hinzu und das Jahr, in dem die Handlung spielt.
\speak{assistant} \leavevmode\\
\speak{assistant} \leavevmode\\
Hier ist eine Tabelle mit den Indiana-Jones-Filmen, sowohl auf Deutsch als auch auf Englisch, sowie den Erscheinungsjahren und den Jahren, in denen die Handlungen der Filme spielen:

\begin{table}[h]
\centering
\small
\begin{tabu} to \linewidth { |X[m,c]|X[m,c]|X[m,c]|X[m,c]| }

\hline
Deutscher Titel & Englischer Titel & Erscheinungsjahr & Jahr der Handlung\\\hline 
\hline
\textbf{Jäger des verlorenen Schatzes} & Raiders of the Lost Ark & 1981 & 1936\\\hline 
\textbf{Indiana Jones und der Tempel des Todes} & Indiana Jones and the Temple of Doom & 1984 & 1935\\\hline 
\textbf{Indiana Jones und der letzte Kreuzzug} & Indiana Jones and the Last Crusade & 1989 & 1938\\\hline 
\textbf{Indiana Jones und das Königreich des Kristallschädels} & Indiana Jones and the Kingdom of the Crystal Skull & 2008 & 1957\\\hline 
\textbf{Indiana Jones und das Rad des Schicksals} & Indiana Jones and the Dial of Destiny & 2023 & 1969\\\hline 
\end{tabu}
\end{table}

Falls du weitere Details oder Informationen benötigst, lass es mich wissen!
\speak{user} \leavevmode\\
Könntest Du für "`Jäger des verlorenen Schatzes"' eine Inhaltszusammenfassung liefern? Bitte nenne die jeweiligen Szenen und Orte der Handlung.
\speak{assistant} \leavevmode\\
Hier ist eine detaillierte Inhaltszusammenfassung von \textbf{„Jäger des verlorenen Schatzes“} (\textbf{„Raiders of the Lost Ark“}), mit Szenen und den dazugehörigen Orten:
\end{dialogue}
\subsection*{1. Einführungsszene – Südamerika, 1936}
Der Film beginnt im Dschungel von Peru. Indiana Jones (Harrison Ford) und sein Begleiter Satipo (Alfred Molina) sind auf der Suche nach einem goldenen Götzen in einem Tempel. Die Szene zeigt den ikonischen Moment, in dem Indy die Statue mit einem Sandsack austauscht und anschließend eine Reihe von tödlichen Fallen auslöst. Nachdem er den Tempel knapp überlebt, wird ihm der Götze von seinem Rivalen René Belloq (Paul Freeman) und einheimischen Kriegern abgenommen.
\begin{itemize}
  \begin{item}
    \textbf{Ort:} Dschungel von Peru (fiktiver Tempel)
  \end{item}
\end{itemize}
\subsection*{2. Rückkehr nach Amerika – Universität, USA}
Zurück in den USA ist Indiana Jones Professor für Archäologie. An der Universität wird er von zwei Agenten der US-Regierung kontaktiert, die ihn über die Aktivitäten der Nazis informieren. Diese suchen nach der \textbf{Bundeslade}, einer uralten religiösen Reliquie, die angeblich unbesiegbare Macht verleiht. Die Nazis arbeiten mit Belloq zusammen und suchen in Ägypten nach der Lade.
\begin{itemize}
  \begin{item}
    \textbf{Ort:} Universität in den USA
  \end{item}
\end{itemize}
\subsection*{3. Nepal – Marion Ravenswood}
Indy reist nach Nepal, um das Medaillon von Marion Ravenwood (Karen Allen) zu finden, das der Schlüssel zur Lokalisierung der Lade ist. Marion ist die Tochter seines verstorbenen Mentors Abner Ravenwood. In Marions Bar in den nepalesischen Bergen treffen sie auf Nazi-Agenten, angeführt von Major Toht. Eine Schießerei bricht aus, und die Bar brennt nieder, doch Indy und Marion entkommen mit dem Medaillon.
\begin{itemize}
  \begin{item}
    \textbf{Ort:} Marions Bar in Nepal
  \end{item}
\end{itemize}
\subsection*{4. Kairo, Ägypten – Suche nach der Lade}
Indy und Marion reisen nach Kairo, wo sie Sallah (John Rhys-Davies) treffen, einen alten Freund und erfahrenen Ausgräber. Sie finden heraus, dass die Nazis in der Nähe von Tanis graben. Indy entdeckt, dass Belloq und die Nazis an der falschen Stelle suchen, weil ihnen eine wichtige Information über die genaue Position der Lade fehlt.
\begin{itemize}
  \begin{item}
    \textbf{Ort:} Kairo und Umgebung, Tanis, Ägypten
  \end{item}
\end{itemize}
\subsection*{5. Die Grube der Seelen – Tanis}
Indy und Sallah gelangen zu einem verborgenen Raum unter der Erde, der als die Grube der Seelen bekannt ist, wo die Lade verborgen sein soll. Mithilfe des Medaillons bestimmen sie die genaue Position der Lade. Indy und Sallah finden die Lade, doch Belloq und die Nazis nehmen sie gefangen. Marion, die zuerst als tot geglaubt wurde, wird ebenfalls gefangen genommen.
\begin{itemize}
  \begin{item}
    \textbf{Ort:} Tanis, Ägypten
  \end{item}
\end{itemize}
\subsection*{6. Verfolgungsjagd – Flucht aus Ägypten}
Indy gelingt es, aus der Gefangenschaft zu entkommen, und er stiehlt einen Lastwagen, auf dem die Nazis die Lade transportieren. Eine packende Verfolgungsjagd quer durch die Wüste beginnt, und Indy kann die Lade in seine Gewalt bringen. Schließlich verstecken Indy, Marion und Sallah die Lade auf einem Schiff.
\begin{itemize}
  \begin{item}
    \textbf{Ort:} Ägyptische Wüste und Hafenstadt
  \end{item}
\end{itemize}
\subsection*{7. U-Boot und Insel der Nazis}
Das Schiff wird von einem deutschen U-Boot abgefangen, und die Lade fällt erneut in die Hände der Nazis. Indy folgt ihnen auf eine entlegene Insel, wo Belloq und die Nazis ein Ritual durchführen wollen, um die Macht der Lade zu entfesseln.
\begin{itemize}
  \begin{item}
    \textbf{Ort:} Insel im Mittelmeer
  \end{item}
\end{itemize}
\subsection*{8. Finale – Öffnung der Lade}
Die Nazis öffnen die Lade in einem religiösen Ritual. Doch die Macht der Lade ist zu groß, und Engel oder Geister kommen hervor. Die Nazis und Belloq werden durch die übernatürliche Kraft der Lade getötet, während Indy und Marion überleben, weil sie die Augen geschlossen halten und nicht hinsehen.
\begin{itemize}
  \begin{item}
    \textbf{Ort:} Insel im Mittelmeer
  \end{item}
\end{itemize}
\subsection*{9. Abschluss – Rückkehr nach Amerika}
Indy und Marion kehren in die USA zurück. Die Lade wird von der US-Regierung beschlagnahmt und in ein riesiges Lagerhaus gebracht, wo sie in einer Kiste verschlossen und versteckt wird.
\begin{itemize}
  \begin{item}
    \textbf{Ort:} Lagerhaus der US-Regierung, USA
  \end{item}
\end{itemize}
\textbf{Zusammenfassung:}\
Indiana Jones reist von den peruanischen Dschungeln nach Nepal und Ägypten, um die Bundeslade zu finden, während er sich mit alten Rivalen, Nazis und tödlichen Fallen auseinandersetzen muss.
\begin{dialogue}
\speak{user} \leavevmode\\
Liefere eine Tabelle, in welcher Szene welche Figuren um's Leben kommen.
\speak{assistant} \leavevmode\\
Hier ist eine Tabelle, die zeigt, welche Figuren in welchen Szenen von \textbf{„Jäger des verlorenen Schatzes“} ums Leben kommen:

\begin{table}[h]
\centering
\small
\begin{tabu} to \linewidth { |X[m,c]|X[m,c]|X[m,c]|X[m,c]| }

\hline
Szene & Ort & Figur(en) & Todesursache\\\hline 
\hline
\textbf{1. Tempelfalle in Peru} & Tempel in Peru & Satipo (Alfred Molina) & Durch eine Falle aufgespießt\\\hline 
\textbf{1. Flucht aus dem Tempel} & Tempel in Peru & Einheimischer Führer & Erschossen von Indiana Jones, als er Indy verrät\\\hline 
\textbf{3. Schießerei in Marions Bar} & Marions Bar, Nepal & Nazis und Handlanger & Erschossen bei der Schießerei in der Bar\\\hline 
\textbf{5. Tanis – Die Grube der Seelen} & Tanis, Ägypten & Verschiedene Arbeiter der Nazis & Von Schlangen gebissen\\\hline 
\textbf{6. Verfolgungsjagd in der Wüste} & Ägyptische Wüste & Verschiedene Nazi-Soldaten & Bei der Verfolgungsjagd erschossen oder von Fahrzeugen überrollt\\\hline 
\textbf{7. U-Boot und Insel der Nazis} & Insel im Mittelmeer & Verschiedene Nazi-Soldaten & Bei Kämpfen und Explosionen getötet\\\hline 
\textbf{8. Öffnung der Bundeslade} & Insel im Mittelmeer & René Belloq (Paul Freeman) & Zerstört durch übernatürliche Kraft der Lade (explodierender Kopf)\\\hline 
 &  & Major Toht (Ronald Lacey) & Durch Geister aus der Lade verbrannt\\\hline 
 &  & Nazi-Kommandant Dietrich (Wolf Kahler) & Zerstört durch die Macht der Lade (Schrumpfen des Körpers)\\\hline 
 &  & Verschiedene Nazi-Soldaten & Durch die Geister und die übernatürliche Macht der Lade getötet\\\hline 
\end{tabu}
\end{table}

Diese Tabelle fasst zusammen, wann und wie die Hauptcharaktere und verschiedene Nebenfiguren während des Films sterben.
\speak{user} \leavevmode\\
Kannst Du die genauen Opferzahlen hinzufügen?
\speak{assistant} \leavevmode\\
Hier ist die erweiterte Tabelle mit den genauen Opferzahlen, wo diese bekannt sind, für \textbf{„Jäger des verlorenen Schatzes“}:

\begin{table}[h]
\centering
\small
\begin{tabu} to \linewidth { |X[m,c]|X[m,c]|X[m,c]|X[m,c]|X[m,c]| }

\hline
Szene & Ort & Figur(en) & Todesursache & Opferzahl\\\hline 
\hline
\textbf{1. Tempelfalle in Peru} & Tempel in Peru & Satipo (Alfred Molina) & Durch eine Falle aufgespießt & 1\\\hline 
\textbf{1. Flucht aus dem Tempel} & Tempel in Peru & Einheimischer Führer & Erschossen von Indiana Jones, als er ihn verrät & 1\\\hline 
\textbf{3. Schießerei in Marions Bar} & Marions Bar, Nepal & Nazis und Handlanger & Erschossen bei der Schießerei in der Bar & 5 (geschätzt)\\\hline 
\textbf{5. Tanis – Die Grube der Seelen} & Tanis, Ägypten & Verschiedene Arbeiter der Nazis & Von Schlangen gebissen & Unbekannt (mind. 2)\\\hline 
\textbf{6. Verfolgungsjagd in der Wüste} & Ägyptische Wüste & Verschiedene Nazi-Soldaten & Bei der Verfolgungsjagd erschossen oder von Fahrzeugen überrollt & 7 (geschätzt)\\\hline 
\textbf{7. U-Boot und Insel der Nazis} & Insel im Mittelmeer & Verschiedene Nazi-Soldaten & Bei Kämpfen und Explosionen getötet & 6 (geschätzt)\\\hline 
\textbf{8. Öffnung der Bundeslade} & Insel im Mittelmeer & René Belloq (Paul Freeman) & Zerstört durch übernatürliche Kraft der Lade (explodierender Kopf) & 1\\\hline 
 &  & Major Toht (Ronald Lacey) & Durch Geister aus der Lade verbrannt & 1\\\hline 
 &  & Nazi-Kommandant Dietrich (Wolf Kahler) & Zerstört durch die Macht der Lade (Schrumpfen des Körpers) & 1\\\hline 
 &  & Verschiedene Nazi-Soldaten & Durch die Geister und die übernatürliche Macht der Lade getötet & Mind. 30\\\hline 
\end{tabu}
\end{table}
\end{dialogue}
\subsection*{Gesamtopferzahl (geschätzt): ~54}
Diese Zahl berücksichtigt Haupt- und Nebenfiguren, wobei genaue Opferzahlen bei größeren Gruppen (z. B. bei der Lade-Szene oder während der Verfolgungsjagd) schwer zu bestimmen sind.


\end{document}

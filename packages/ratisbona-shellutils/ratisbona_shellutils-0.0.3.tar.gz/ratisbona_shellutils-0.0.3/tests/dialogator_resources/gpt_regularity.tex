\documentclass[a4paper]{article}
\usepackage[ngerman]{babel}
\usepackage[dvipsnames]{xcolor}
\usepackage[export]{adjustbox} % loads also graphicx
\usepackage{amsmath}
\usepackage{emoji}
\usepackage{fontspec}
\usepackage{dialogue}
\usepackage{wasysym}
\usepackage{minted}
\usepackage{cprotect}
\usepackage[colorlinks=true, allcolors=TealBlue]{hyperref}
\usepackage{longtable,tabu}
\setlength{\parindent}{0ex}
\setlength{\parskip}{1ex}
\setminted{breaklines=true, breakanywhere=true}

\newcommand{\uliji}[1]{
  {\setmainfont{Noto Color Emoji}[Renderer=Harfbuzz]{#1}}
}


\author{Ulrich Schwenk}
\title{test gpt}

\begin{document}
\setlength{\tabulinesep}{6pt}
\maketitle
\newpage

\begin{dialogue}
\speak{system} \leavevmode\\
\end{dialogue}




\section*{2025-02-15}
\begin{dialogue}
\speak{system} \leavevmode\\
\speak{user} \leavevmode\\
Wie genau unterscheidet man in der Informatik, welche Funktionalität man in den Parser und welche man in den Tokenizer packt? In welchem Zusammenhang steht das mit dem Begriff "`Regularität"'?
\speak{assistant} \leavevmode\\
In der Informatik, insbesondere in der Compiler- und Sprachverarbeitung, unterscheidet man zwischen \textbf{Tokenizer (Lexikalische Analyse)} und \textbf{Parser (Syntaktische Analyse)} anhand der \textbf{Art der Grammatik}, die zur Beschreibung der Sprache verwendet wird. Der zentrale Begriff, der dabei hilft, die Grenze zwischen den beiden Komponenten zu ziehen, ist die \textbf{Regularität}.

\end{dialogue}
\subsection*{1. Tokenizer (Lexikalische Analyse)}
Der Tokenizer (auch \textbf{Lexer} oder Scanner genannt) ist für die \textbf{Zerlegung eines Eingabetextes in Tokens} zuständig. Dabei geht es um \textbf{Wortbildung} und \textbf{Struktur auf der Ebene regulärer Sprachen}. Ein Tokenizer:
\begin{itemize}
  \begin{item}
    \textbf{Erkennt reguläre Muster} in einem Eingabetext (z. B. Schlüsselwörter, Identifikatoren, Zahlen, Operatoren).
  \end{item}
  \begin{item}
    \textbf{Arbeitet mit regulären Ausdrücken} oder \textbf{endlichen Automaten}, um Zeichenketten zu gruppieren.
  \end{item}
  \begin{item}
    Hat keine Hierarchie oder rekursive Strukturen, sondern betrachtet nur aufeinanderfolgende Zeichenfolgen.
  \end{item}
\end{itemize}
Beispiel für Tokenizer-Regeln in einer Programmiersprache:

\begin{minted}{text}
IF -> Schlüsselwort
x  -> Bezeichner
=  -> Zuweisungsoperator
42 -> Ganzzahl
;  -> Semikolon
\end{minted}
Der Tokenizer gibt dann eine Sequenz von Tokens aus, z. B.:

\begin{minted}{text}
[IF, IDENT(x), ASSIGN_OP, INT(42), SEMICOLON]
\end{minted}
Ein \textbf{entscheidendes Merkmal} ist, dass ein Tokenizer \textbf{keine Kontextfreiheit benötigt}. Er verarbeitet die Eingabe \textbf{rein sequentiell} nach regulären Mustern.

\uliji{➡} \textbf{Tokenization basiert auf regulären Sprachen} (Regularität!).
\subsection*{2. Parser (Syntaktische Analyse)}
Der Parser analysiert die \textbf{hierarchische Struktur} der Token-Sequenz nach den Regeln einer \textbf{kontextfreien Grammatik} (CFG, Context-Free Grammar). Er überprüft, ob die Reihenfolge der Tokens mit der zugrunde liegenden Sprache übereinstimmt.

Ein Parser:
\begin{itemize}
  \begin{item}
    \textbf{Liest Tokens ein und prüft die Syntax} nach einer \textbf{kontextfreien Grammatik}.
  \end{item}
  \begin{item}
    \textbf{Erkennt verschachtelte Strukturen} (z. B. verschachtelte Klammern, Funktionsaufrufe).
  \end{item}
  \begin{item}
    Nutzt \textbf{Parsebäume} oder \textbf{Ableitungsregeln}, um zu entscheiden, ob eine Token-Sequenz gültig ist.
  \end{item}
\end{itemize}
Beispiel für eine kontextfreie Grammatik für eine Zuweisung:

\begin{minted}{text}
<statement> ::= <identifier> "=" <expression> ";"
<expression> ::= <number> | <identifier>
\end{minted}
Diese Regel erlaubt z. B. die Eingabe:

\begin{minted}{text}
x = 42;
\end{minted}
Der Parser würde die Token-Sequenz \mintinline{text}{[IDENT(x), ASSIGN_OP, INT(42), SEMICOLON]} analysieren und bestätigen, dass sie zur Regel \mintinline{text}{<statement>} passt.

\uliji{➡} \textbf{Parsing erfordert kontextfreie Grammatiken, die über Regularität hinausgehen}.
\subsection*{3. Zusammenhang mit "`Regularität"'}
Der Begriff \textbf{Regularität} bezieht sich auf die \textbf{Klassen von Grammatiken} in der Chomsky-Hierarchie:

\begin{table}[h]
\centering
\small
\begin{tabu} to \linewidth { |X[m,c]|X[m,c]|X[m,c]| }

\hline
\textbf{Grammatikkategorie} & \textbf{Beispiel} & \textbf{Verwendung}\\\hline 
\hline
\textbf{Reguläre Sprachen (Regular Expressions, RE)} & Keywords, Zahlen, Operatoren & \textbf{Tokenization} (durch reguläre Ausdrücke oder endliche Automaten)\\\hline 
\textbf{Kontextfreie Sprachen (Context-Free Grammar, CFG)} & Verschachtelte Strukturen (z. B. \mintinline{text}{if (x == 1) { ... }}) & \textbf{Parsing} (durch Parser wie LL(1), LR(1), etc.)\\\hline 
\end{tabu}
\end{table}

\textbf{Die zentrale Regel:}
\begin{itemize}
  \begin{item}
    \textbf{Alles, was mit regulären Sprachen beschrieben werden kann, gehört in den Tokenizer.}
  \end{item}
  \begin{item}
    \textbf{Alles, was kontextfreie (oder sogar kontextabhängige) Strukturen benötigt, gehört in den Parser.}
  \end{item}
\end{itemize}
\subsection*{4. Konkrete Unterscheidung zwischen Tokenizer und Parser}
\subsubsection*{\uliji{✅} Gehört in den Tokenizer (Lexer)?}
\begin{itemize}
  \begin{item}
    \textbf{Ja, wenn es mit regulären Ausdrücken beschrieben werden kann}:
    \begin{itemize}
      \begin{item}
        Schlüsselwörter (\mintinline{text}{if}, \mintinline{text}{while}, \mintinline{text}{return})
      \end{item}
      \begin{item}
        Bezeichner (\mintinline{text}{myVar}, \mintinline{text}{x})
      \end{item}
      \begin{item}
        Zahlen (\mintinline{text}{42}, \mintinline{text}{3.14})
      \end{item}
      \begin{item}
        Operatoren (\mintinline{text}{+}, \mintinline{text}{-}, \mintinline{text}{*})
      \end{item}
      \begin{item}
        Klammern (\mintinline{text}{(}, \mintinline{text}{)})
      \end{item}
    \end{itemize}
  \end{item}
\end{itemize}
\subsubsection*{\uliji{✅} Gehört in den Parser?}
\begin{itemize}
  \begin{item}
    \textbf{Ja, wenn eine hierarchische, rekursive Struktur vorhanden ist}:
    \begin{itemize}
      \begin{item}
        Verschachtelte Ausdrücke (\mintinline{text}{(a + (b * c))})
      \end{item}
      \begin{item}
        Bedingte Anweisungen (\mintinline{text}{if (x == 1) { y = 2; }})
      \end{item}
      \begin{item}
        Schleifen (\mintinline{text}{while (x < 10) { x++; }})
      \end{item}
    \end{itemize}
  \end{item}
\end{itemize}
\subsection*{5. Beispiel zur Abgrenzung}
Angenommen, wir analysieren folgenden Code:

\begin{minted}{c}
if (x == 10) { y = 5; }
\end{minted}
\begin{itemize}
  \begin{item}
    \textbf{Tokenizer erzeugt Tokens:}

\begin{minted}{text}
[IF, LPAREN, IDENT(x), EQ_OP, INT(10), RPAREN, LBRACE, IDENT(y), ASSIGN_OP, INT(5), SEMICOLON, RBRACE]
\end{minted}
  \end{item}
  \begin{item}
    \textbf{Parser überprüft die Struktur} anhand einer Grammatik:

\begin{minted}{text}
<if_statement> ::= "if" "(" <condition> ")" "{" <statement> "}"
\end{minted}
  \end{item}
\end{itemize}
\uliji{💡} \textbf{Schlussfolgerung:}
\begin{itemize}
  \begin{item}
    Der \textbf{Tokenization-Prozess ist regulär} (basiert auf festen Mustern).
  \end{item}
  \begin{item}
    Die \textbf{Parsing-Phase erfordert eine kontextfreie Grammatik}, weil sie \textbf{Verschachtelung} und \textbf{rekursive Strukturen} erkennt.
  \end{item}
\end{itemize}
\subsection*{6. Warum kann der Tokenizer nicht alles erledigen?}
Weil reguläre Sprachen \textbf{keine rekursiven Strukturen} ausdrücken können! Ein regulärer Ausdruck kann z. B. nicht erkennen, ob eine Klammerstruktur korrekt ist:

\begin{minted}{text}
((()))  ✅ korrekt
(()))   ❌ inkorrekt
\end{minted}
\uliji{➡} \textbf{Hier braucht man einen Parser}, weil Klammerpaare eine \textbf{kontextfreie Grammatik} benötigen.
\subsection*{7. Fazit}
\begin{itemize}
  \begin{item}
    \textbf{Der Tokenizer bearbeitet reguläre Strukturen} (durch \textbf{reguläre Ausdrücke} oder \textbf{endliche Automaten}).
  \end{item}
  \begin{item}
    \textbf{Der Parser bearbeitet rekursive, kontextfreie Strukturen} (mithilfe einer \textbf{kontextfreien Grammatik}).
  \end{item}
  \begin{item}
    \textbf{Regularität} trennt \textbf{lexikalische Analyse (Tokenizer)} von der \textbf{syntaktischen Analyse (Parser)}, da reguläre Sprachen nicht ausreichen, um verschachtelte Strukturen zu analysieren.
  \end{item}
\end{itemize}
\uliji{🔹} \textbf{Merksatz:}\leavevmode\\\uliji{📝} \textbf{„Was mit regulären Ausdrücken beschreibbar ist, gehört in den Tokenizer. Alles mit verschachtelten Strukturen gehört in den Parser.“}


\end{document}
